

\section{Processes: shades of openness}

\subsection{Open Process}

\begin{frame}{What is Open Process}{}

\begin{itemize}
\item one could begin with large brush strokes - in doing so we would lay the foundations or the background colours which make everything else possible for the project to emerge: open process

\item The most important attributes of the development of the Internet, the Web and their communication-cooperation 
  tools is openness of the entire process of production.

\item ``Publishing and knowledge production in academia can be significantly improved if aspects of cooperative models developed in software and networking communities are adopted.''  
\end{itemize}
\end{frame}

\begin{frame}{Benefits of open process}{}

\begin{itemize}
\item increase in the quality of submissions
\item increase in the quality and innovation of published text
\item faster and more responsive pace of research 
\item attracting more risk taking and innovative contributors 
\item gain readership and reputation 
\end{itemize}
\end{frame}

\begin{frame}{Benefits for internal processes}{}
\begin{itemize}
\item recognition of the most active and important workers (i.e. no free riders)
\item decision making in the hands of those who do the most work, more transparently
\item easier and simplified project management
\item attract new volunteers 
\item reduce the impact of counter-productive participants
\end{itemize}

\end{frame}


\section{journal commons}

\subsection{Platform choice}

\begin{frame}{Zope/Plone}{}
\begin{description}

\item [mature] the web platform we are proposing to use is Zope/Plone.
	      a very mature web platform (ten years old)
\item [CMS] integrated, i.e. one single website
\item [workflows] at the root of all this
\item [platform] i/vcal, rss, WebDAV, etc.
\end{description}
\end{frame}  


\subsection{Architecture}{}

\begin{frame}{Objects}{journalcommons.Journal}
 \begin{description}
  \item [Journal] This object is a container for all other objects plus
  settings and bibliographical data of the journal 
  \item [Issue] A container for published issues
  \mycomment{Issue: the basic object, entity, of the CSA journal will be a 'issue' - which functions as a way of creating clusters of work
 
A text can be submitted to the journal only as part of an issue, not individually
 
Plone allows for us to programme an object called [issue], which can be created by any of the CSA members, or by a chosen group of editors, internally on the website (the inside site) 
 
Once an [issue] object (imagine it just as a folder which stores submissions - Plone treats it as a folder with special properties) is created, it becomes a sub-section on the website, in the section Journal 
 
Access to it can be fine-tuned for example: a) only editors can view it; b) editors + authors who submit; c) all CSA members; d) full public view 
 
JOURNAL -> ISSUE -> ARTICLE -> PRINTED COLLECTION}


  \item [Article] Drafts, Comments (referee, EB/CE), history
  \item [plus]  Section, EditorsMeeting, Portlets, etc.
  
 \end{description}

\end{frame}

\begin{frame}{Objects}{journalcommons.Conference}
 \begin{description}
  \item [Conference] a journal
  \item [Paper] a journal
  \item [Event] a journal
 \end{description}
\end{frame}

\begin{frame}{Objects}{journalcommons.Lab}

 \begin{description}
  \item [Thread] a journal
  \mycomment{Time-line of publishing and printed formats
 
articles are published on the 'outside' site whenever they are ready
there is no pressure to lower standards or to rush the process in order to meet a deadline
at the same time the journal is able to cope with a faster and more responsive pace of research 
 
Issue type 1 - Research Threads: 2-3 years deadlines. Continuous online publishing, as soon as peer reviews are done and editors decide on the article

Issue type 2 - Special Issues: commissioned work, organized as standard commisioned or CfP special issues. Any form of reviewing can be chosen. It can also be used as a part of a Research Thread.}


  \item [Paper] explain 
 \end{description}
\end{frame}

%
%
%
\begin{frame}{Workflows}{Peer Reviewed}
\mycomment{workflows tell us what happens when a submission is made - all possible steps through which a submission can go through in order to be accepted, rejected, or signaled (categories with options - Whitworth \& Friedman, 2009, First Monday). 
 
Submission of projects and articles will be done via the website Each time the state of an article changes, authors will be automatically informed via email, while the details (article text, issue name, etc) are all automatically filled in by the software 
 
We will be able to see queues of articles in different stages (submitted, being reviewed by issue editors, being peer reviewed, accepted, rejected, signaled, etc) in boxes.
 
Each member can have a customised view of the entire website for example - an issues editor will be able to see only portlets (queues with articles in stages) to do with their issue.
\bigskip
{\bf Talk about Conference, plus variations}
}


\begin{center}
\includegraphics[width=3.5in]{WorkflowPR.png}
\end{center}

\end{frame}




\begin{frame}{Workflows}{Non Peer Reviewed}
\begin{center}
\includegraphics[width=2.7in]{WorkflowNonPR.png}
\end{center}

\end{frame}


\begin{frame}{Submissions}{Track your work}

\begin{center}
\includegraphics[width=4in]{1-conference-submissions-homepage.jpg}
\end{center}
  
\mycomment{Explain that Submissions come a bit from Poi (Issue Tracker, 
  Trouble Tickets) by status, by action, by responsible, by alarms (3 months...)
  }
\end{frame} 

\begin{frame}{Submissions}{Track your work}
\begin{center}
\includegraphics[width=4in]{2-conference-SubmitConferencePaper-form.jpg}
\end{center}
\end{frame} 

\begin{frame}{Submissions}{Track your work}
\begin{center}
\includegraphics[width=3.5in]{3-conference-ListAllSubmissions.jpg}
\end{center}
\end{frame} 

\begin{frame}{Submissions}{Track your work}
\begin{center}
\includegraphics[width=4in]{4-conference-SubmissionsCategoriesAndActions.jpg}
\end{center}
\end{frame} 

\section{Shades of openness}

\begin{frame}{Contiuum of practices}
 
\begin{itemize}
\item Since whole process is on a website. Then {\bf opennes==ACLs}

\item We can accommodate everything from blind peer review, extend to CE/AB or open process (open process reviewing, early screening). 
Even publishing of arts works with some form of peer review (curating or collective models).

\item We can do it all simultaneously, using one collaborative-reviewing-publishing platform

\item We can even display articles/artsworks published through those different types of workflows with an icon (so that they can appear next to each other clearly distinguished) signifying: blind-peer-review, open-process peer review, semi-open-process, artwork-peer-reviewed, artwork-curated, etc. (PlaceableWorkflow and acquisition)
\end{itemize}
\end{frame}

\begin{frame}{Signals}

Signals are a way to choose a series of drop-down menu ratings, signaling to the reader on several aspects of the article, rather then using the binary publish/reject model. Fuzzy logic.

\begin{description}
\item [types] activist, academic, journalistic
 \mycomment{activist (article proposes a critique of a policy or practice with specific 
action proposals or suggestions), academic (article follows conventions of academic research article, position in literature, cited sources, and claimed contribution)}

\item [language quality] expression/narrative of article 
\item [logical flow] ideas are well organised in article
\item [originality] the argument presented in article is new
\item [evidence] there are many established arguments for which the most valuable contribution would be further and better evidence
\item [commendations] signal appreciation of the article. brief statement rather than a drop-down menu with options --- 50 words.
\mycomment{Or perhaps recommended to others 1 (only to those with a very specific interest) to 10 (universal essential knowledge)}
\end{description}

\end{frame}

\section{Further directions}
\begin{frame}{journalcommons}{}
 
\begin{itemize}
 \item {\tt www.journal-commons.org}
 \item Conference
 \item Books: book publishing platform
\end{itemize}
 
\end{frame}
